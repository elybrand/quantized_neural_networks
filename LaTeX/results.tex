\documentclass[journal,onecolumn,11pt,final]{IEEEtran}
\usepackage[utf8]{inputenc}
\usepackage[TS1,T1]{fontenc}
\usepackage{fourier, heuristica}
\usepackage{array, booktabs}
\usepackage{graphicx}
\usepackage{grffile}
\usepackage[x11names]{xcolor}
\usepackage{colortbl}
\usepackage{caption}
\DeclareCaptionFont{blue}{\color{LightSteelBlue3}}
\usepackage[T1]{fontenc}
%\usepackage[latin9]{inputenc}
%\usepackage{babel}
\usepackage{float}
\usepackage{amsthm}
\usepackage{amsmath}
\usepackage{amssymb}
\usepackage{dsfont}
\usepackage{tcolorbox}
\usepackage[unicode=true,
 bookmarks=true,bookmarksnumbered=true,bookmarksopen=true,bookmarksopenlevel=1,
 breaklinks=false,pdfborder={0 0 0},backref=false,colorlinks=false]
 {hyperref}
\hypersetup{pdftitle={Your Title},
 pdfauthor={Your Name},
 pdfpagelayout=OneColumn,pdfnewwindow=true,pdfstartview=XYZ,plainpages=false}
 
 
\newcommand{\sd}{\Sigma\Delta}
\newcommand{\ds}{\displaystyle}
\newcommand{\1}{\mathbb{1}}
\newcommand{\one}{\mathbbm{1}}
\newcommand{\R}{\mathbb{R}}
\newcommand{\F}{\mathbb{F}}
\newcommand{\BPDN}{\rm{BPDN}}
\newcommand{\x}{\mathbf{x}}
\newcommand{\X}{\mathbf{X}}
\newcommand{\dd}{\mathop{}\!\mathrm{d}}
\newcommand{\TPR}{\text{TPR}}
\newcommand{\A}{\mathcal{A}}
\newcommand{\N}{\mathbb{N}}
\newcommand{\eps}{\varepsilon}
\renewcommand{\Re}{\text{Re}}
\renewcommand{\Im}{\text{Im}}
%\newcomannd{\red}
\makeatletter

\DeclareMathOperator{\diag}{diag}
\DeclareMathOperator{\antidiag}{antidiag}
\DeclareMathOperator{\supp}{supp}
\DeclareMathOperator{\rank}{rank}
\DeclareMathOperator{\tr}{tr}
\DeclareMathOperator{\dist}{dist}
\DeclareMathOperator*{\argmax}{arg\,max}    % nice argmax operator
\DeclareMathOperator*{\argmin}{arg\,min}    % nice argmin operator
\DeclareMathOperator*{\sign}{sign}    % nice argmin operator


%%%%%%%%%%%%%%%%%%%%%%%%%%%%%% LyX specific LaTeX commands.
%% Because html converters don't know tabularnewline
% \providecommand{\tabularnewline}{\\}
% \floatstyle{ruled}
% \newfloat{algorithm}{tbp}{loa}
% \providecommand{\algorithmname}{Algorithm}
% \floatname{algorithm}{\protect\algorithmname}

% %%%%%%%%%%%%%%%%%%%%%%%%%%%%%% Textclass specific LaTeX commands.
%  % protect \markboth against an old bug reintroduced in babel >= 3.8g
%  \let\oldforeign@language\foreign@language
%  \DeclareRobustCommand{\foreign@language}[1]{%
%    \lowercase{\oldforeign@language{#1}}}
\theoremstyle{plain}
\newtheorem{thm}{\protect\theoremname}
\newtheorem{prop}[thm]{\protect\propname}
\newtheorem{lem}[thm]{Lemma}
\theoremstyle{definition}
\newtheorem{defn}[thm]{\protect\definitionname}
\newtheorem{assumption}{\protect\assumptionname}
\newtheorem{remark}{Remark}
\theoremstyle{plain}
\theoremstyle{plain}
\newtheorem{cor}[thm]{\protect\corollaryname}


\newtheorem{theorem}{Theorem}[section]
\newtheorem{pro}[theorem]{Proposition}
\newtheorem{lemma}[theorem]{Lemma}
\newtheorem{fact}[theorem]{Fact}
\newtheorem{conjecture}[theorem]{Conjecture}
\newtheorem{claim}[theorem]{Claim}
\newtheorem{corollary}[theorem]{Corollary}
\theoremstyle{definition}
\newtheorem{definition}[theorem]{Definition}
\newtheorem{example}[theorem]{Example}
\newtheorem{xca}[theorem]{Exercise}

%%%%%%%%%%%%%%%%%%%%%%%%%%%%%% User specified LaTeX commands.
%\usepackage{babel}
% for subfigures/subtables




\providecommand{\corollaryname}{Corollary}
\providecommand{\propname}{Proposition}
\providecommand{\definitionname}{Definition}
\providecommand{\theoremname}{Theorem}
\providecommand{\assumptionname}{Assumption}

\makeatother

\providecommand{\corollaryname}{Corollary}
\providecommand{\definitionname}{Definition}
\providecommand{\theoremname}{Theorem}

\newcommand{\ELnote}[1]{\textcolor{red}{[{\em {\bf **EL Note:} #1}]}}
\newcommand{\red}[1]{\textcolor{red}{#1}}
\newcommand{\blue}[1]{\textcolor{blue}{#1}}


\renewcommand{\A}{\mathbb{A}}
\newcommand{\B}{\mathbb{B}}
\renewcommand{\C}{\mathbb{C}}
\newcommand{\D}{\mathbb{D}}
\newcommand{\E}{\mathbb{E}}
\renewcommand{\F}{\mathbb{F}}
%\renewcommand{\G}{\mathbb{G}}
\renewcommand{\H}{\mathbb{H}}
\newcommand{\I}{\mathbb{I}}
\newcommand{\J}{\mathbb{J}}
\newcommand{\K}{\mathbb{K}}
\renewcommand{\L}{\mathbb{L}}
\newcommand{\M}{\mathbb{M}}
\renewcommand{\N}{\mathbb{N}}
\renewcommand{\O}{\mathbb{O}}
\renewcommand{\P}{\mathbb{P}}
\newcommand{\Q}{\mathbb{Q}}
\renewcommand{\R}{\mathbb{R}}
\renewcommand{\S}{\mathbb{S}}
\newcommand{\T}{\mathbb{T}}
\renewcommand{\U}{\mathbb{U}}
\newcommand{\V}{\mathbb{V}}
\newcommand{\W}{\mathbb{W}}
\renewcommand{\X}{\mathbb{X}}
\newcommand{\Y}{\mathbb{Y}}
\newcommand{\Z}{\mathbb{Z}}

\newcommand{\Ac}{\mathcal{A}}
\newcommand{\Bc}{\mathcal{B}}
\newcommand{\Cc}{\mathcal{C}}
\newcommand{\Dc}{\mathcal{D}}
\newcommand{\Ec}{\mathcal{E}}
\newcommand{\Fc}{\mathcal{F}}
\newcommand{\Gc}{\mathcal{G}}
\newcommand{\Hc}{\mathcal{H}}
\newcommand{\Ic}{\mathcal{I}}
\newcommand{\Jc}{\mathcal{J}}
\newcommand{\Kc}{\mathcal{K}}
\newcommand{\Lc}{\mathcal{L}}
\newcommand{\Mc}{\mathcal{M}}
\newcommand{\Nc}{\mathcal{N}}
\newcommand{\Oc}{\mathcal{O}}
\newcommand{\Pc}{\mathcal{P}}
\newcommand{\Qc}{\mathcal{Q}}
\newcommand{\Rc}{\mathcal{R}}
\newcommand{\Sc}{\mathcal{S}}
\newcommand{\Tc}{\mathcal{T}}
\newcommand{\Uc}{\mathcal{U}}
\newcommand{\Vc}{\mathcal{V}}
\newcommand{\Wc}{\mathcal{W}}
\newcommand{\Xc}{\mathcal{X}}
\newcommand{\Yc}{\mathcal{Y}}
\newcommand{\Zc}{\mathcal{Z}}



\begin{document}
%%%%%%%%%%%%%%%%%%%%%%%%%%%%%%%%%%%%%%%%%%%%%%%%%%%%%%%%%%%%%%%%%
\begin{center}
	{\LARGE Bag of Results}
\end{center}
%\smallskip
\hrule height 1pt 

%%%%%%%%%%%%%%%%%%%%%%%%%%%%%%%%%%%%%%%%%%%%%%%%%%%%%%%%%%%%%%%%%
\vspace{10pt}
\noindent
%
%\begin{lemma}\label{lem: big jump neg incr}
%	If \(|w_t - q_t| > 1\), then \((w_t-q_t)^2 + 2(w_t-q_t)X_t^Tu_{t-1} < |w_t| \min\{-1-w_t, -1+w_t\}\). \ELnote{Please double check that it's \(-|w_t| (1+|w_t|)\)}
%\end{lemma}
%\begin{proof}
%	
%	Recall that \(q_t = \Qc(w_t + X_t^T u_{t-1})\). Consider the case when \(w_t - q_t > 1\). By virtue of \(w_t \leq 1\), it must be the case that \(q_t = -1\). Consequentially, we have \(w_t > 0\), and \(X_t^Tu_{t-1} < -1/2 - w_t\). Using these equalities and inequalities, we find \((w_t - q_t)^2 + 2(w_t-q_t)X_t^Tu_{t-1} < -w_t (w_t + 1) < 0\). An analogous argument holds in the case when \(q_t - w_t > 1\).
%\end{proof}
%
%\begin{lemma}
%	Let \(|w_t| < 1\), \(\|X_t\|_2 \leq 1\). Then \((w_t - q_t)^2 + 2(w_t-q_t)X_t^Tu_{t-1} \leq 1/4\) for all \(i > 0\).
%\end{lemma}
%\begin{proof}
%	Intuitively, one can reason this by noting that the case when \(X_t^Tu_{t-1} = 0\) is ``the worst''. If such were the case, then the worst error you could incur would be \((w_t - q_t)^2 = 1/4\).  Let us proceed more formally. We may consider three separate cases. The first case, when \(|w_t - q_t| > 1\), is handled in Lemma \ref{lem: big jump neg incr}.
%	
%	Now, consider when \(w_t - q_t \in (1/2, 1)\). Note that \(q_t = 1\) is not possible, for this would violate \(w_t < 1\).  If \(q_t = 0\), then \(w_t \in (1/2, 1)\), \(X_t^Tu_{t-1} \in (-1/2 - w_t, 1/2-w_t)\), and so \((w_t - q_t)^2 + 2(w_t-q_t)X_t^Tu_{t-1} \in (-w_t - w_t^2, w_t - w_t^2)\), where we remark that \(w_t - w_t^2 \leq 1/4\) for \(w_t \in (1/2, 1)\). If it is the case instead that \(q_t = -1\), then \(w_t \in (-1/2, 0)\), \(X_t^Tu_{t-1} < -1/2 - w_t\), and so \((w_t - q_t)^2 + 2(w_t-q_t)X_t^Tu_{t-1} \leq |w_t| (1 - |w_t|) < 0\). \ELnote{Check that an analogous case holds when \(q_t - w_t \in (1/2, 1)\)}.
%	
%	Finally, we consider when \(w_t - q_t < 1/2.\) \ELnote{TODO}
%\end{proof}

\begin{lemma}\label{lem: q rounds w with dither}	
	\begin{align*}
		q_t = \Qc\left( w_t + \frac{X_t^T u_{t-1}}{\|X_t\|_2^2}\right).
	\end{align*}
\end{lemma}
\begin{proof}
	This follows simply by completing a square. Provided \(X_t \neq 0\), we have by the definition of \(q_t\)
	\begin{align*}
		q_t &= \argmin_{p \in \{-1, 0, 1\}} \| u_{t-1} + (w_t - p) X_t\|_2^2 = \argmin_{p \in \{-1, 0, 1\}} (w_t - p)^2 + 2(w_t-p) \frac{X_t^T u_{t-1}}{\|X_{t-1}\|_2^2}\\
		&=  \argmin_{p \in \{-1, 0, 1\}} \left((w_t - p) +  \frac{X_t^T u_{t-1}}{\|X_{t-1}\|_2^2}\right)^2 -  \left(\frac{X_t^T u_{t-1}}{\|X_{t-1}\|_2^2}\right)^2.
	\end{align*}
	Because the former term is always non-negative, it must be the case that the minimizer is \( \Qc\left( w_t + \frac{X_t^T u_{t-1}}{\|X_t\|_2^2}\right)\).
\end{proof}

\begin{lemma}\label{lem: 1-cdf increments, pos w}
	Suppose \( 0 < w_t < 1\), and \(X_t\) a random vector in \(R^d\). Then
	\begin{align*}
		\P_{X_t}\left( (w_t - q_t)^2 + 2(w_t - q_t)X_t^T u_{t-1} > \alpha \Big| u_{t-1} \right) = \left\{
			\begin{array}{ll}
			      \mu_y\left( \frac{\alpha - (w_t + 1)^2}{2(w_t + 1)}, \frac{\alpha - (w_t - 1)^2}{2(w_t - 1)} \right) & \alpha < -w_t - w_t^2 \\
			      \mu_y\left( \frac{\alpha - w_t^2}{2w_t}, \frac{\alpha - (w_t - 1)^2}{2(w_t - 1)} \right) &  -w_t - w_t^2 \leq \alpha \leq w_t - w_t^2\\
			      0 & \alpha > w_t - w_t^2
			\end{array} 
			\right. 
	\end{align*}
	where \(\mu_y\) is the probability measure over \(\R\) induced by the random variable \(y = X_t^T u_{t-1}\).
\end{lemma}
\begin{proof}
	Let \(A_b\) denote the event that \(q_t = b\) for \(b \in \{-1, 0, 1\}\). Then by the law of total probability
	\begin{align*}
	\P\left( (w_t - q_t)^2 + 2(w_t - q_t)y > \alpha \Big| u_{t-1} \right) = \sum_{b \in \{-1, 0, 1\}} \P\left( (w_t - b)^2 + 2(w_t - b)y > \alpha \Big| u_{t-1}, A_b \right).
	\end{align*}
	\begin{itemize}
		\item \textbf{b = 0} \(q_t = 0\) precisely when \(-1/2 - w_t \leq  y \leq 1/2 - w_t\). So we have
		\begin{align*}
		&\P\left( w_t^2 + 2w_t y > \alpha \Big| u_{t-1}, A_b \right) = \P\left( y > \frac{\alpha-w_t^2}{2w_t} \Big| u_{t-1}, -1/2 - w_t \leq y \leq 1/2-w_t\right) \\
		&=\left \{ \begin{array}{ll}
			      \mu_y\left( -1/2 - w_t, 1/2 - w_t\right) & \alpha < -w_t - w_t^2 \\
			        \mu_y\left( \frac{\alpha-w_t^2}{2w_t}, 1/2 - w_t\right) &  -w_t - w_t^2 \leq \alpha \leq w_t - w_t^2\\
			      0 & \alpha > w_t - w_t^2
			\end{array} .
			\right.  
		\end{align*}
		
		\item \textbf{b = 1} \(q_t = 1\) precisely when \( y > 1/2 - w_t\). Noting that \(w_t - 1 < 0\), we have
		\begin{align*}
		&\P\left( (w_t - 1)^2 + 2(w_t - 1)y > \alpha \Big| u_{t-1}, A_1 \right) = \P\left( y < \frac{\alpha-(w_t - 1)^2}{2(w_t-1)} \Big| u_{t-1}, y > 1/2 - w_t \right) \\		
		&=\left\{ \begin{array}{ll}
			        \mu_y\left( 1/2 - w_t,  \frac{\alpha - (w_t - 1)^2}{2(w_t - 1)} \right) &   \alpha \leq w_t - w_t^2\\
			      0 & \alpha > w_t - w_t^2
			\end{array} .
			\right.  		
			\end{align*}
			
		\item \textbf{b = -1} \(q_t = -1\) precisely when \( y < -1/2 - w_t\). So we have
		\begin{align*}
		&\P\left( (w_t + 1)^2 + 2(w_t - 1)y > \alpha \Big| u_{t-1}, A_1 \right) = \P\left( y > \frac{\alpha-(w_t + 1)^2}{2(w_t+1)} \Big| u_{t-1}, y < -1/2 - w_t \right) \\		
		&=\left\{ \begin{array}{ll}
			        \mu_y\left( \frac{\alpha - (w_t + 1)^2}{2(w_t + 1)}, -1/2 - w_t \right) &   \alpha \leq -w_t - w_t^2\\
			      0 & \alpha > -w_t - w_t^2
			\end{array} .
			\right.  		
			\end{align*}
	\end{itemize}
 	Summing these three functions yields the result.
\end{proof}

\begin{corollary}\label{corr: 1-cdf increments, neg w}
	When \(-1 < w_t < 0\), we have
	\begin{align*}
		\P_{X_t}\left( (w_t - q_t)^2 + 2(w_t - q_t)X_t^T u_{t-1} > \alpha \Big| u_{t-1} \right) = \left\{
			\begin{array}{ll}
			      \mu_y\left( \frac{\alpha - (w_t + 1)^2}{2(w_t + 1)}, \frac{\alpha - (w_t - 1)^2}{2(w_t - 1)} \right) & \alpha < w_t - w_t^2 \\
			      \mu_y\left( \frac{\alpha - (w_t + 1)^2}{2(w_t + 1)}, \frac{\alpha - w_t^2}{2w_t} \right) &  w_t - w_t^2 \leq \alpha \leq -w_t - w_t^2\\
			      0 & \alpha > w_t - w_t^2
			\end{array} .
			\right. 
	\end{align*}
\end{corollary}

%\begin{corollary}
%	\begin{align*}
%		\P_{X_t}\left( (w_t - q_t)^2 + 2(w_t - q_t)X_t^T u_{t-1} > w_t^2 - |w_t | \Big| u_{t-1} \right) = \mu_y(-1/2, 1/2).
%	\end{align*}
%\end{corollary}
%
%\begin{lemma}\label{lem: cdf absolute increments}
%	Suppose \( 0 < w_t < 1\), and \(X_t\) a random vector in \(R^d\). Then
%	\begin{align*}
%		\P_{X_t}\left( \Big| \Delta \|u_t\|_2^2 \Big| \leq \alpha \Big| u_{t-1} \right) = \left\{
%			\begin{array}{ll}
%			     \mu_y\left( \frac{-\alpha - w_t^2 }{2w_t}, \frac{\alpha - w_t^2}{2w_t} \right) +  \mu_y\left( \frac{-\alpha + (w_t - 1)^2}{2(1-w_t)}, \frac{\alpha + (w_t - 1)^2}{2(1 - w_t)} \right) & 0 \leq \alpha < w_t - w_t^2 \\
%			      \mu_y\left( \frac{-\alpha - w_t^2}{2w_t}, \frac{\alpha + (w_t - 1)^2}{2(1-w_t)} \right) &  w_t - w_t^2 \leq \alpha \leq w_t + w_t^2\\
%			       \mu_y\left( \frac{-\alpha - (w_t+1)^2}{2(w_t+1)}, \frac{\alpha + (w_t - 1)^2}{2(1-w_t)} \right) & \alpha > w_t + w_t^2
%			\end{array} 
%			\right. 
%	\end{align*}
%	where \(\mu_y\) is the probability measure over \(\R\) induced by the random variable \(y = X_t^T u_{t-1}\).
%\end{lemma}

\begin{lemma}\label{lem: expected qX}
	Let \(X_t \sim \mathrm{Unif}(S^{d-1})\). Then
	\begin{align*}
		\E[q_t X_t | u_{t-1}] = \frac{\mathrm{Area}(S^{d-2})}{(d-1) \cdot \mathrm{Area}(S^{d-1})}  \left(\left(1 - \min\left\{\left(\frac{1/2 - w_t}{\|u_{t-1}\|}\right)^2, 1\right\} \right)^{\frac{d-1}{2}} + \left(1 - \min\left\{\left(\frac{1/2 + w_t}{\|u_{t-1}\|}\right)^2, 1\right\} \right)^{\frac{d-1}{2}} \right) \frac{u_{t-1}}{\|u_{t-1}\|_2},
	\end{align*}
	where \(\mu_y\) is the probability measure over \(\R\) induced by the random variable \(y = X_t^T u_{t-1}\).
\end{lemma}
\begin{proof}
	By unitary invariance of the \(X_t\), we may assume that \(u_{t-1} = \|u_{t-1}\|_2 e_1\). Under this assumption, we have
	\begin{align*}
		\E[q_t X_t | u_{t-1}] = \E[\Qc\left(w_t + \|u_{t-1}\|_2 X_{t,1} \right) X_t | u_{t-1}].
	\end{align*}
	Breaking this apart into the events \(A_{-1} := \{ X_{t,1} \leq \frac{-1/2 - w_t}{\|u_{t-1}\|} \}\) and \(A_{1} := \{ X_{t,1} \geq \frac{1/2 - w_t}{\|u_{t-1}\|} \}\), we note
	that these regions on \(S^{d-1}\) are symmetric about the \(u_{t-1} = e_1\) axis. So,
	\begin{align*}
		\E[q_t X_t | u_{t-1}] = \E[X_t | u_{t-1}, A_1] - \E[X_t | u_{t-1}, A_{-1}].
	\end{align*}
	Using the hyperspherical coordinates
	\begin{align}\label{eq: hyperspherical coordinates}
		f\left(\begin{bmatrix} \varphi_1 \\ \varphi_2 \\ \vdots \\ \varphi_{d-1} \end{bmatrix}\right) = \begin{bmatrix} \cos(\varphi_1) \\ \sin(\varphi_1)\cos(\varphi_2) \\ \sin(\varphi_1)\sin(\varphi_2)\cos(\varphi_3) \\ \vdots \\ \sin(\varphi_1)\hdots \sin(\varphi_{d-2}) \cos(\varphi_{d-1}) \\  \sin(\varphi_1)\hdots \sin(\varphi_{d-2}) \sin(\varphi_{d-1})\end{bmatrix},
	\end{align}
	we have \(\E[X_{t,1} | u_{t-1}, A_1] = 0\) if \(\frac{1/2 - w_t}{\|u_{t-1}\|_2} > 1\). Otherwise,
	\begin{align*}
		 \E[X_{t,1} | u_{t-1}, A_1] &= \frac{1}{\mathrm{Area}(S^{d-1})} \int_{0}^{\arccos(\frac{1/2 - w_t}{\|u_{t-1}\|_2})} \cos(\varphi_1) dA\\
		 &= \frac{\mathrm{Area}(S^{d-2})}{\mathrm{Area}(S^{d-1})} \int_{0}^{\arccos(\frac{1/2 - w_t}{\|u_{t-1}\|_2})} \cos(\varphi_1) \sin^{d-2}(\varphi_1) d\varphi_1 =  \frac{\mathrm{Area}(S^{d-2})}{(d-1)\mathrm{Area}(S^{d-1})} \left(1-\left(\frac{1/2 - w_t}{\|u_{t-1}\|_2}\right)^2\right)^{\frac{d-1}{2}}.
	\end{align*}
	Similarly, \(\E[X_{t,1} | u_{t-1}, A_1] = 0\) if \(\frac{1/2 - w_t}{\|u_{t-1}\|_2} \leq -1\). Otherwise,
	\begin{align*}
		\E[X_{t,1} | u_{t-1}, A_1] =  \frac{\mathrm{Area}(S^{d-2})}{(d-1)\mathrm{Area}(S^{d-1})} \left(1-\left(\frac{1/2 + w_t}{\|u_{t-1}\|_2}\right)^2\right)^{\frac{d-1}{2}}.
	\end{align*}
%	Using the ``method of cross-sections'' to reduce to a univariate integral, we find that we only need to average the first component of \(X_t\) in the expectations by symmetry. Hence,
%	\begin{align*}
%	 & \E[X_{t,1} | u_{t-1}, A_1] = \frac{1}{\mathrm{Area}(S^{d-1})} \int_{\min\left\{\frac{1/2 - w_t}{\|u_{t-1}\|_2},1\right\}}^1 x \cdot \mathrm{vol}\left(\sqrt{1-x^2} S^{d-2}\right)\,\,dx 
%		%	 
%		= \frac{\mathrm{Area}(S^{d-2})}{\mathrm{Area}(S^{d-1})}  \int_{\min\left\{\frac{1/2 - w_t}{\|u_{t-1}\|_2},1\right\}}^1 x \cdot (1-x^2)^{\frac{d-1}{2}}\,\,dx \\
%		%
%	 &= \frac{\mathrm{Area}(S^{d-2})}{(1+d)\mathrm{Area}(S^{d-1})}  \left(1 - \min\left\{\left(\frac{1/2 - w_t}{\|u_{t-1}\|}\right)^2, 1\right\} \right)^{\frac{d+1}{2}}
%	\end{align*}
%	Similarly, 
%	\begin{align*}
%		 &\E[X_{t,1} | u_{t-1}, A_{-1}] =  \frac{-\mathrm{Area}(S^{d-2})}{(1+d)\mathrm{Area}(S^{d-1})}\left(1 - \min\left\{\left(\frac{1/2 + w_t}{\|u_{t-1}\|}\right)^2, 1\right\} \right)^{\frac{d+1}{2}}
%	\end{align*}
\end{proof}

\begin{lemma} \label{lem: ratio of spherical areas}
	Let \(d \geq 2\). Then 
	\begin{align*}
		\sqrt{ \frac{d}{4e^3} }. \leq  \frac{\mathrm{Area}(S^{d-2})}{\mathrm{Area}(S^{d-1})} \leq \frac{e}{\pi} \sqrt{d}
	\end{align*}
\end{lemma}
\begin{proof}
	Recall that the Stirling bounds on the Gamma function give
	\begin{align*}
		\sqrt{2\pi} n^{n+1/2} e^{-n} \leq \Gamma(n) \leq e n^{n+1/2} e^{-n}
	\end{align*}
	for all \(n \in \N\). Using these bounds along with the fact that \(\mathrm{Area}(S^{d-1}) = \frac{d \pi^{d/2}}{\Gamma\left( \frac{d}{2} + 1\right)}\), we have for the lower bound \ELnote{Of course, it can never be the case that both \( \frac{d}{2}\) and \(\frac{d-1}{2}\) are both integers. This seems like a small, annoying detail though.}
	\begin{align*}
		\frac{\mathrm{Area}(S^{d-2})}{\mathrm{Area}(S^{d-1})} 
		%
		&= \frac{d-1}{d} \frac{1}{\sqrt{\pi}} \frac{\Gamma\left( \frac{d}{2} + 1\right)}{\Gamma\left( \frac{d-1}{2} + 1\right)}
		%
		\geq \frac{1}{2} \frac{\sqrt{2\pi}}{e\sqrt{\pi}} \frac{(\frac{d+2}{2})^{\frac{d+3}{2}} e^{-\frac{d+2}{2}}}{(\frac{d+1}{2})^{\frac{d+2}{2}}e^\frac{-d+1}{2}}
		%
		=  \frac{1}{\sqrt{2e^3}} \frac{(\frac{d+2}{2})^{\frac{d+3}{2}}}{(\frac{d+1}{2})^{\frac{d+2}{2}}}\\
		%
		&= \sqrt{\frac{d+2}{4e^3}} \left(\frac{d+2}{d+1}\right)^{\frac{d+2}{2}} 
		%
		\geq \sqrt{\frac{d}{4e^3}} .
	\end{align*}
	Now, for the upper bound,
	\begin{align*}
		\frac{\mathrm{Area}(S^{d-2})}{\mathrm{Area}(S^{d-1})}
		%
		&= \frac{d-1}{d} \frac{1}{\sqrt{\pi}} \frac{\Gamma\left( \frac{d}{2} + 1\right)}{\Gamma\left( \frac{d-1}{2} + 1\right)}
		%
		\leq \frac{1}{\sqrt{\pi}} \frac{e \left(\frac{d+2}{2} \right)^{\frac{d+3}{2}} e^{- \,  \frac{d+2}{2} }}{\sqrt{2\pi} \left( \frac{d+1}{2} \right)^{\frac{d+2}{2}} e^{- \, \frac{d+1}{2}}}
		%
		= \frac{\sqrt{e}}{\pi \sqrt{2}} \left( \frac{d+2}{2} \right)^{\frac{1}{2}} \left( \frac{d+2}{d+1} \right)^{\frac{d+2}{2}}\\
		%
		&\leq  \frac{\sqrt{e}}{\pi \sqrt{2}} \sqrt{d} \left( 1 + \frac{1}{d+1} \right)^{\frac{d+1}{2}} \cdot  \left( 1 + \frac{1}{d+1} \right)^{\frac{1}{2}}
		%
		\leq \frac{e}{\pi} \sqrt{d}.
	\end{align*}
\end{proof}

\begin{lemma} \label{lem: conditionally supermartingale}
	For any \(\eps > 0\) and \(C_{\eps} =  \frac{(4+\eps)e^{21/8}}{2}\), we have
	\begin{align*}
		\E\left[\|u_t\|_2^2 - \|u_{t-1}\|_2^2 \Big| \|u_{t-1}\|_2^2 > C^2 (d-1) \right] \leq -\eps.
	\end{align*}
\end{lemma}
\begin{proof}
	Let \(\eps > 0\). For any fixed \(u_{t-1}\) with \(\|u_{t-1}\|_2 > C_{\eps}\sqrt{d-1}\), we have by Lemma \ref{lem: expected qX} and Lemma \ref{lem: ratio of spherical areas}
	\begin{align*}
		&\E\left[\|u_t\|_2^2 - \|u_{t-1}\|_2^2 \Big| u_{t-1} \right] = \E\left[ (w_t - q_t)^2 + 2(w_t - q_t)X_t^T u_{t-1}  \Big| u_{t-1} \right] \leq 4 - 2 \E\left[q_tX_t^T u_{t-1}  \Big| u_{t-1}\right]\\
		%
		&= 4 - 2 \|u_{t-1}\|_2 \frac{\mathrm{Area}(S^{d-2})}{(d-1) \cdot \mathrm{Area}(S^{d-1})}  \left(\left(1 - \left(\frac{1/2 - w_t}{\|u_{t-1}\|}\right)^2 \right)^{(d-1)/2} + \left(1 - \left(\frac{1/2 + w_t}{\|u_{t-1}\|}\right)^2 \right)^{(d-1)/2} \right)\\
		%
		&\leq 4 - 2 C_{\eps}\frac{\mathrm{Area}(S^{d-2})}{\sqrt{d}\mathrm{Area}(S^{d-1})}  \left(\left(1 - \frac{(1/2 - w_t)^2}{d-1} \right)^{(d-1)/2} + \left(1 - \frac{(1/2 + w_t)^2}{d} \right)^{(d-1)/2} \right)\\
		%
		&\leq 4 - 2 C_{\eps}\frac{\mathrm{Area}(S^{d-2})}{\sqrt{d}\mathrm{Area}(S^{d-1})}  \left(\exp\left( - \frac{(1/2-w_t)^2}{2}\right) + \exp\left( - \frac{(1/2+w_t)^2}{2} \right)\right)\\
		%
		&\leq 4 - 2C_{\eps} \sqrt{\frac{1}{4e^3}} \left(\exp\left( - \frac{(1/2-w_t)^2}{2}\right) + \exp\left( - \frac{(1/2+w_t)^2}{2} \right)\right) \\
		%
		& \leq 4 - 4 C_{\eps}e^{-\frac{9}{8}} \sqrt{\frac{1}{4e^3}} = 4 - \frac{4 C_{\eps}}{2e^{21/8}} = -\eps < 0.
	\end{align*}
\end{proof}

\begin{lemma} \label{lem: mgf increments}
	Let \(w_t\) be fixed. If \(X_t \sim \mathrm{Unif}(S^{d-1})\), then for any \(\lambda \in (0,1)\) and \(\beta > 2\)
	\begin{align*}
		\E\left[e^{\lambda \Delta \|u_{t}\|_2^2} \Big| \|u_{t-1}\|_2 \geq \beta \right] \leq  \frac{e\sqrt{d}}{\beta\pi} \cdot \frac{\max\left\{e^{w_t^2}, e^{(1-w_t)^2}, e^{(1+w_t)^2}\right\}^{\lambda}}{2\lambda \min\{w_t, 1-w_t, 1+w_t\}} \Big( 
				e^{-2\lambda(1-w_t)(1/2-w_t)} + e^{-2\lambda(w_t+1)(1/2+w_t)} + e^{2\lambda w_t (1/2-w_t)} 
		\Big).
	\end{align*}
\end{lemma}
\begin{proof}
		As we usually have done, we'll split the conditional expectation up into the three events which determine the values of \(q\). Note that whenever \(\|u_{t-1}\| \geq 2\), each event has non-zero probability of occurring regardless of \(w_t\) (this is why we assume \(\beta > 2\)). By rotational invariance, we may assume that \(u_{t-1} = \|u_{t-1}\|_2 e_1\). Using the hyperspherical coordinates as in \eqref{eq: hyperspherical coordinates}, we have
		\begin{align*}
			\E\left[e^{\lambda \Delta \|u_{t}\|_2^2} \Big| \|u_{t-1}\|_2 \geq \beta \right] = 
			%
			\frac{\mathrm{Area}(S^{d-2})}{\mathrm{Area}(S^{d-1})} \Big( &e^{\lambda (w_t - 1)^2}\int_{\arccos(1)}^{\arccos\left(\frac{1/2-w_t}{\|u_{t-1}\|}\right)}	e^{2\lambda (w_t - 1) \|u_{t-1}\|_2 \cos(\varphi)} \sin^{d-2}(\varphi) \,\, d\varphi \\
			%
			+ &e^{\lambda w_t^2} \int_{\arccos\left(\frac{1/2 - w_t}{\|u_{t-1}\|_2}\right)}^{\arccos\left(\frac{-1/2-w_t}{\|u_{t-1}\|}\right)} e^{2\lambda w_t \|u_{t-1}\| \cos(\varphi)} \sin^{d-2}(\varphi) \,\, d\varphi \\
%	
			+ &e^{\lambda (w_t + 1)^2} \int_{\arccos\left(\frac{-1/2 - w_t}{\|u_{t-1}\|_2}\right)}^{\arccos(-1)} e^{2\lambda (w_t + 1) \|u_{t-1}\|_2 \cos(\varphi)} \sin^{d-2}(\varphi)\,\,d\varphi \Big).
		\end{align*}
		We'll control each integral individually. Note for the first in the sum of three, we may decompose it as
		\begin{align*}
			&\int_{\arccos(1)}^{\arccos\left(\frac{1/2-w_t}{\|u_{t-1}\|}\right)}	e^{2\lambda (w_t - 1) \|u_{t-1}\|_2 \cos(\varphi)} \sin^{d-2}(\varphi) \,\, d\varphi  \\
			%
			&= \int_{0}^{\frac{\pi}{2}}	e^{2\lambda (w_t - 1) \|u_{t-1}\|_2 \cos(\varphi)} \sin^{d-2}(\varphi) \,\, d\varphi 
			%
			+ \int_{\frac{\pi}{2}}^{ \arccos\left(\frac{1/2-w_t}{\|u_{t-1}\|}\right)}	e^{2\lambda (w_t - 1) \|u_{t-1}\|_2 \cos(\varphi)} \sin^{d-2}(\varphi) \,\, d\varphi \\
			%
			&\leq \int_{0}^{\frac{\pi}{2}} \sin^{d-2}(\varphi) \,\, d\varphi + \left| \arccos\left( \frac{1/2-w_t}{\|u_{t-1}\|} \right) - \arccos\left( 0 \right) \right|\max_{\varphi \in \left[\pi/2, \arccos\left(\frac{1/2-w_t}{\|u_{t-1}\|_2}\right)\right]} e^{2\lambda (w_t - 1) \|u_{t-1}\|_2 \cos(\varphi)}\\
			%
			&\leq \int_{0}^{\frac{\pi}{2}} \sin^{d-2}(\varphi) \,\, d\varphi 
			%
			+ \max_{|x| \leq  \frac{1/2 - w_t}{\|u_{t-1}\|_2}} \left| \frac{1}{\sqrt{1-x^2}} \right|  \frac{1}{\|u_{t-1}\|_2} \max\left\{e^{2\lambda (w_t - 1)(1/2-w_t)}, 1\right\}\\
			%
			&\leq \frac{\mathrm{Area}(S^{d-1})}{2\mathrm{Area}(S^{d-2})} + \frac{16}{7\|u_{t-1}\|_2}  \max\left\{e^{2\lambda (w_t - 1)(1/2-w_t)}, 1\right\} \leq \red{\frac{\mathrm{Area}(S^{d-1})}{2\mathrm{Area}(S^{d-2})}} + \frac{16}{7\beta} \max\left\{e^{2\lambda (w_t - 1)(1/2-w_t)}, 1\right\}.
		\end{align*}
		The third integral follows a similar argument, namely
		\begin{align*}
			&\int_{\arccos\left(\frac{-1/2 - w_t}{\|u_{t-1}\|_2}\right)}^{\arccos(-1)} e^{2\lambda (w_t + 1) \|u_{t-1}\|_2 \cos(\varphi)} \sin^{d-2}(\varphi)\,\,d\varphi \\
			%
			& \leq \int_{\arccos\left(\frac{-1/2 - w_t}{\|u_{t-1}\|_2}\right)}^{\frac{\pi}{2}} e^{2\lambda (w_t + 1) \|u_{t-1}\|_2 \cos(\varphi)} \sin^{d-2}(\varphi)\,\,d\varphi + \int_{\frac{\pi}{2}}^{\pi} \sin^{d-2}(\varphi)\,\,d\varphi\\
			%
			&\leq \frac{16}{7\beta}  \max\left\{e^{2\lambda (w_t - 1)(-1/2-w_t)}, 1\right\} + \red{\frac{\mathrm{Area}(S^{d-1})}{2\mathrm{Area}(S^{d-2})}}.
		\end{align*}
		Finally, for the second integral we use the same upper bound + mean value theorem trick as we've used twice above
		\begin{align*}
			& \int_{\arccos\left(\frac{1/2 - w_t}{\|u_{t-1}\|_2}\right)}^{\arccos\left(\frac{-1/2-w_t}{\|u_{t-1}\|}\right)} e^{2\lambda w_t \|u_{t-1}\| \cos(\varphi)} \sin^{d-2}(\varphi) \,\, d\varphi\\
			%
			&= \int_{\arccos\left(\frac{1/2 - w_t}{\|u_{t-1}\|_2}\right)}^{\frac{\pi}{2}} e^{2\lambda w_t \|u_{t-1}\| \cos(\varphi)} \sin^{d-2}(\varphi) \,\, d\varphi + \int_{\frac{\pi}{2}}^{\arccos\left(\frac{-1/2-w_t}{\|u_{t-1}\|}\right)} e^{2\lambda w_t \|u_{t-1}\| \cos(\varphi)} \sin^{d-2}(\varphi) \,\, d\varphi\\
			%
			&\leq \frac{32}{7\beta} \max\left\{e^{2\lambda w_t (1/2 - w_t)} , e^{2\lambda w_t (-1/2 - w_t)}, 1\right\}.
		\end{align*}
		Combining these three upper bounds give
		\begin{align*}
			&\E\left[e^{\lambda \Delta \|u_{t}\|_2^2} \Big| \|u_{t-1}\|_2 \geq \beta \right]
			%
			&\leq \max\left\{e^{w_t^2}, e^{(1+w_t)^2} \right\} \frac{\mathrm{Area}(S^{d-2})}{\mathrm{Area}(S^{d-1})} \Big( \frac{\mathrm{Area}(S^{d-1})}{\mathrm{Area}(S^{d-2})}
		\end{align*}
		\ELnote{This new proof also seems to be very sensitive to how close \(w_t\) is to \(\pm 1\). If you can't modify this version to get it to work, uncomment out the block below and use it instead.		
		On second thought, this is entirely unsurprising. \(\Delta \|u_{t-1}\|_2^2 = 0\) if \(w_t = 0, \pm1\), so the m.g.f can't be bounded away from the origin in these cases.}
%		
%		These integrals are challenging to compute when \(d-2 > 1\), but since these integrands are always nonnegative for \(\varphi \in [0, \pi]\), we can always use the (quite loose) upper bound \(\sin^{d-2}(\varphi) \leq \sin(\varphi)\). Using this in the above integrals, we may now easily evaluate these dominating integrals using substitution methods to get
%		\begin{align*}
%			\E\left[e^{\lambda \Delta \|u_{t}\|_2^2} \Big| \|u_{t-1}\|_2 \geq \beta \right] \leq 
%		     %
%		     \frac{\mathrm{Area}(S^{d-2})}{\mathrm{Area}(S^{d-1})} \Big( 
%		     	&e^{\lambda (w_t - 1)^2} \cdot \frac{-e^{2\lambda (w_t-1) \|u_{t-1}\|} + e^{2\lambda (w_t-1)(1/2 - w_t)}}{2\lambda \|u_{t-1}\|(1-w_t)} \\
%		     	&+ e^{\lambda w_t^2} \frac{e^{2\lambda w_t (1/2-w_t)} - e^{-2\lambda w_t (1/2 + w_t)}}{2\lambda w_t \|u_{t-1}\|}\\
%		     	&+ e^{\lambda (w_t + 1)^2} \frac{e^{-2\lambda(w_t+1)(1/2+w_t)} - e^{-2\lambda (w_t+1)\|u_{t-1}\|}}{2\lambda (1+w_t)\|u_{t-1}\|}
%		     \Big).
%		\end{align*}
%		\begin{align*}
%		\leq \frac{\mathrm{Area}(S^{d-2})}{\beta\mathrm{Area}(S^{d-1})} \cdot \frac{ \max\left\{e^{w_t^2}, e^{(1-w_t)^2}, e^{(1+w_t)^2}\right\}^{\lambda}}{2\lambda \min\{w_t, 1-w_t, 1+w_t\}} \Big( 
%				&-e^{-2\lambda(w_t+1)\|u_{t-1}\|} - e^{-2\lambda(1-w_t)\|u_{t-1}\|}\\
%				&+ e^{-2\lambda(1-w_t)(1/2-w_t)} + e^{-2\lambda(w_t+1)(1/2+w_t)}\\
%				&+ e^{2\lambda w_t (1/2-w_t)} - e^{-2\lambda w_t(1/2+w_t)}
%		\Big).
%		\end{align*}
%		Now, if \(\lambda > 0\) we can upper-bound this quantity by
%				\begin{align*}
%		&\leq \frac{\mathrm{Area}(S^{d-2})}{\beta\mathrm{Area}(S^{d-1})} \cdot \frac{\max\left\{e^{w_t^2}, e^{(1-w_t)^2}, e^{(1+w_t)^2}\right\}^{\lambda}}{2\lambda \min\{w_t, 1-w_t, 1+w_t\}} \Big( 
%				e^{-2\lambda(1-w_t)(1/2-w_t)} + e^{-2\lambda(w_t+1)(1/2+w_t)} + e^{2\lambda w_t (1/2-w_t)} 
%		\Big)\\
%		&\leq \frac{e\sqrt{d}}{\beta\pi} \cdot \frac{\max\left\{e^{w_t^2}, e^{(1-w_t)^2}, e^{(1+w_t)^2}\right\}^{\lambda}}{2\lambda \min\{w_t, 1-w_t, 1+w_t\}} \Big( 
%				e^{-2\lambda(1-w_t)(1/2-w_t)} + e^{-2\lambda(w_t+1)(1/2+w_t)} + e^{2\lambda w_t (1/2-w_t)} 
%		\Big).
%		\end{align*}
\end{proof}

\begin{theorem}
	For any \( t \in \N\), we have
	\begin{align*}
		\P\left( \|u_{t}\|_2^2 > \alpha \right) \leq ...
	\end{align*}
\end{theorem}
\begin{proof}
	The proof essentially follows the argument in the main result of Hajek's work \cite{,,} using Lemma \ref{lem: mgf increments} to, in the notation of Hajek, secure condition D1 and the fact that \(\E\left[e^{\lambda \Delta \|u_{t}\|_2^2} \Big| \|u_{t-1}\|_2^2 \right] \leq e^{\frac{\lambda}{4}}\) in place of condition D2. For the sake of completeness, we include that argument here. Using the familiar Laplace transform trick with Chebyshev's inequality, we have for any \(\lambda > 0\)
	\begin{align*}
		\P\left( \|u_t\|_2^2 \geq \alpha \right) &\leq e^{-\lambda \alpha} \E\left[ e^{\|u_t\|_2^2} \right]
		%
		\leq e^{-\lambda \alpha} \E\left[ e^{\lambda \|u_{t-1}\|_2^2} e^{\lambda \Delta\|u_{t}\|_2^2} \right]
		%
		=  e^{-\lambda \alpha} \E\left[ \E\left[e^{\lambda \|u_{t-1}\|_2^2} e^{\lambda \Delta\|u_{t}\|_2^2} \Big| \Fc_{t-1} \right] \right].
	\end{align*}
	We can now slip this conditional expectation of the moment generating function of the increments into two pieces based on the norm of the previous residual as
	\begin{align*}
		\E\left[e^{\lambda \|u_{t-1}\|_2^2} e^{\lambda \Delta\|u_{t}\|_2^2} \Big| \Fc_{t-1} \right] = \E\left[e^{\lambda \|u_{t-1}\|_2^2} e^{\lambda \Delta\|u_{t}\|_2^2} \Big| \Fc_{t-1}, \|u_{t-1}\|_2 \geq \beta \right] + \E\left[e^{\lambda \|u_{t-1}\|_2^2}e^{\lambda \Delta\|u_{t}\|_2^2} \Big|\Fc_{t-1}, \|u_{t-1}\|_2 < \beta \right]
	\end{align*}
	Specifically, we note by Lemma \ref{lem: mgf increments} when \ELnote{\(\beta \gtrsim \sqrt{d}\)} we have
	\begin{align}\label{eq: condition D1}
		\E\left[ e^{\lambda \Delta\|u_t\|_2^2} \Big| \Fc_{t-1}, \red{\|u_{t-1}\|_2^2 \geq \beta} \right] = \text{\ELnote{ some quantity...}} := \rho < 1.
	\end{align}
	Moreover, by Lemma \ref{lem: 1-cdf increments, pos w} and Corollary \ref{corr: 1-cdf increments, neg w}
	\begin{align}\label{eq: condition D2}
		\E\left[ e^{\lambda\|u_{t-1}\|_2^2} e^{\lambda \Delta\|u_t\|_2^2} \Big| \Fc_{t-1}, \red{\|u_{t-1}\|_2^2 < \beta} \right] \leq e^{\frac{\lambda}{4}} e^{\lambda \beta}.
	\end{align}
	Combining \eqref{eq: condition D1}, \eqref{eq: condition D2} then gives
	\begin{align*}
		\P\left( \|u_t\|_2^2 \geq \alpha \right) \leq e^{-\lambda \alpha}\left( \rho\E\left[ e^{\lambda \|u_{t-1}\|_2^2} \right] + e^{\frac{\lambda}{4}}\right).
	\end{align*}
	Proceeding inductively thus yields
	\begin{align*}
		\P\left( \|u_t\|_2^2 \geq \alpha \right) \leq \rho^t e^{\lambda \left(\|u_0\|_2^2 - \alpha \right)} + \frac{1-\rho^t}{1-\rho} \red{e^{\lambda \left(\beta + \frac{1}{4} - \alpha \right)}}.
	\end{align*}
	\ELnote{Go through and choose \(\beta, \lambda\) appropriately.}
\end{proof}

\end{document}